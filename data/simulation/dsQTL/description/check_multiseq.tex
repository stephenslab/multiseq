% Created 2016-02-13 Sat 13:55
\documentclass[11pt]{article}
\usepackage[utf8]{inputenc}
\usepackage[T1]{fontenc}
\usepackage{fixltx2e}
\usepackage{graphicx}
\usepackage{longtable}
\usepackage{float}
\usepackage{wrapfig}
\usepackage{rotating}
\usepackage[normalem]{ulem}
\usepackage{amsmath}
\usepackage{textcomp}
\usepackage{marvosym}
\usepackage{wasysym}
\usepackage{amssymb}
\usepackage{hyperref}
\tolerance=1000
\usepackage{listings}
\usepackage[usenames,dvipsnames]{color}
\usepackage{underscore}
\usepackage{parskip}
\usepackage{lmodern}
\usepackage{parskip}
\usepackage{graphicx}
\setlength{\parindent}{0pt}
\usepackage{underscore}
\textwidth 16cm
\oddsidemargin 0.5cm
\evensidemargin 0.5cm
\author{Heejung Shim}
\date{\today}
\title{"A simple testing procedure for multiseq"}
\hypersetup{
  pdfkeywords={},
  pdfsubject={},
  pdfcreator={Emacs 24.3.1 (Org mode 8.2.10)}}
\begin{document}

\maketitle
\tableofcontents


\section{Overview}
\label{sec-1}
\begin{itemize}
\item After making changes in a software package multiseq, we run a set of tests to check if the changes in multiseq affect output as we planned (e.g., changes shouldn't affect other outputs).
\item We simulate four sets of data - typical data and data with no reads with/without library read depth.
\end{itemize}

\section{Simulation of data set}
\label{sec-2}
\subsection{Collect information to simulate data}
\label{sec-2-1}
I'll copy phenotype data (from which we will sample data) and signals for group 1 and group 2. See compareWaveQTLandmultiseq\_dsQTL\_final.org for detailed description of how to obtain this information. 
\lstset{breaklines=true,showspaces=false,showtabs=false,tabsize=2,basicstyle=\ttfamily,frame=single,keywordstyle=\color{Blue},stringstyle=\color{BrickRed},commentstyle=\color{ForestGreen},columns=fullflexible,language=bash,label= ,caption= ,numbers=none}
\begin{lstlisting}
cd ~/multiseq/
mkdir data
cd data/
mkdir simulation
cd simulation/
mkdir dsQTL
cp ~/multiscale_analysis/analysis/simulation/sample_size/simulation_QTLfinal_v2/data/pheno.dat dsQTL
cp ~/multiscale_analysis/analysis/simulation/sample_size/simulation_QTLfinal_v2/data/alt.sig0 dsQTL
cp ~/multiscale_analysis/analysis/simulation/sample_size/simulation_QTLfinal_v2/data/alt.sig1 dsQTL
\end{lstlisting}

I'll copy library read depth for this DNase-seq data (you can download from \href{https://github.com/heejungshim/WaveQTL/blob/master/data/dsQTL/library.read.depth.dat}{WaveQTL repo}) to this directory.  
\lstset{breaklines=true,showspaces=false,showtabs=false,tabsize=2,basicstyle=\ttfamily,frame=single,keywordstyle=\color{Blue},stringstyle=\color{BrickRed},commentstyle=\color{ForestGreen},columns=fullflexible,language=bash,label= ,caption= ,numbers=none}
\begin{lstlisting}
cp ~/WaveQTL/data/dsQTL/library.read.depth.dat ~/multiseq/data/simulation/dsQTL/
\end{lstlisting}

Information have been save in: 
\lstset{breaklines=true,showspaces=false,showtabs=false,tabsize=2,basicstyle=\ttfamily,frame=single,keywordstyle=\color{Blue},stringstyle=\color{BrickRed},commentstyle=\color{ForestGreen},columns=fullflexible,language=bash,label= ,caption= ,numbers=none}
\begin{lstlisting}
cd ~/multiseq/data/simulation/dsQTL
alt.sig0  alt.sig1  library.read.depth.dat  pheno.dat
\end{lstlisting}

\subsection{Simulate data}
\label{sec-2-2}
The function `simulate.data' is a modification \href{https://github.com/heejungshim/multiscale_analysis/blob/master/src/R/simulateDataForAll.run.multiscale.R}{of the script in multiscale\_analysis repo}. This script uses functions in `my.utils.R' in \href{https://github.com/heejungshim/multiscale_analysis/}{multiscale\_analysis repo}.
\lstset{breaklines=true,showspaces=false,showtabs=false,tabsize=2,basicstyle=\ttfamily,frame=single,keywordstyle=\color{Blue},stringstyle=\color{BrickRed},commentstyle=\color{ForestGreen},columns=fullflexible,language=R,label= ,caption= ,numbers=none}
\begin{lstlisting}
setwd("~/multiseq/data/simulation/dsQTL/")
source("~/multiscale_analysis/src/R/my.utils.R")
##' `simulate.data' simulate data sets by resampling reads from real data (real.read.counts) for given signals (sig0, sig1).
##' 
##' @param seed seed number to set up
##' @param numGroup0 number of samples for Group0
##' @param numGroup1 number of samples for Group1
##' @param real.read.counts a vector of size T (e.g., 1024); t-th element contains number of reads at t-th position; from which we will resample reads for simulation;
##' @param sig0 a vector of size T (e.g., 1024); t-th element contains probability of sampling read at t-th position in real.read.counts for Group0
##' @param sig1 a vector of size T (e.g., 1024); t-th element contains probability of sampling read at t-th position in real.read.counts for Group1
##' @param real.library.read.depth a vector of size M (>1); from whch we will sample library read depth
##' @param over.dispersion parameter used in sample.from.Binomial.with.Overdispersion; see `my.utils.R' for details. 
##' @return a list of data, group, library.read.depth; data contains simulated data; matrix of (numGroup0+numGroup1) by T; group contains group indicator for simulated data; a vector of size (numGroup0+numGroup1); library.read.depth contains simulated library read depth; a vector of size (numGroup0+numGroup1)
simulate.data <- function(seed = 1, numGroup0, numGroup1, sig0, sig1, real.read.counts, real.library.read.depth = NULL, over.dispersion=NULL){

  genoD = c(rep(0, numGroup0), rep(1, numGroup1))
  numSam = length(genoD)
  numBPs = length(sig0)

  ## phenotype data
  phenoD = matrix(data=NA, nr= length(genoD), nc = numBPs)

  ## let's sample!!!
  set.seed(seed)

  ## upper and lower bound!
  trunc.fun = function(x){
    x = max(0, x)
    return(min(1,x))
  }
  p.sig0 = sapply(sig0, trunc.fun)
  p.sig1 = sapply(sig1, trunc.fun)

  ## geno = 0
  wh0 = which(genoD == 0)
  if(length(wh0) > 0){
    phenoD[wh0,] = sample.from.Binomial.with.Overdispersion(num.sam = length(wh0), total.count = real.read.counts, mu.sig = p.sig0, over.dispersion = over.dispersion)
  }  
  ## geno = 1
  wh1 = which(genoD == 1)
  if(length(wh1) > 0){
    phenoD[wh1,] = sample.from.Binomial.with.Overdispersion(num.sam = length(wh1), total.count = real.read.counts, mu.sig = p.sig1, over.dispersion = over.dispersion)
  }
  if(is.null(real.library.read.depth)){
    library.read.depth=NULL
  }else{
    library.read.depth = sample(real.library.read.depth, numSam, replace = TRUE)
  }
  return(list(data = phenoD, group = genoD, library.read.depth = library.read.depth))
}

seed = 1
numGroup0 = 10
numGroup1 = 10
sig0 = scan("~/multiseq/data/simulation/dsQTL/alt.sig0", what=double())
sig1 = scan("~/multiseq/data/simulation/dsQTL/alt.sig1", what=double())
real.DNase.dat = read.table("~/multiseq/data/simulation/dsQTL/pheno.dat", as.is = TRUE)
real.read.counts = ceiling(as.numeric(apply(real.DNase.dat, 2, sum)))
real.library.read.depth = scan("~/multiseq/data/simulation/dsQTL/library.read.depth.dat", what=double())

## data with sample size 20 with library read depth 
res = simulate.data(seed = seed, numGroup0 = numGroup0, numGroup1 = numGroup1, sig0 = sig0, sig1= sig1, real.read.counts = real.read.counts, real.library.read.depth = real.library.read.depth, over.dispersion=NULL)
str(res)
## List of 3
## $ data              : int [1:20, 1:1024] 0 0 0 0 0 0 0 0 0 0 ...
## $ group             : num [1:20] 0 0 0 0 0 0 0 0 0 0 ...
## $ library.read.depth: num [1:20] 44257311 37331440 30843655 36823292 50966695 ...
apply(res$data,1,sum)
## [1] 65 60 63 62 81 75 57 65 57 66 39 40 49 46 44 48 53 43 37 51

## data with sample size 20 with library read depth, but all samples in group 2 have zero read count.
res = simulate.data(seed = seed, numGroup0 = numGroup0, numGroup1 = numGroup1, sig0 = sig0, sig1= rep(0, length(sig1)), real.read.counts = real.read.counts, real.library.read.depth = real.library.read.depth, over.dispersion=NULL)
apply(res$data,1, sum)
## [1] 65 60 63 62 81 75 57 65 57 66  0  0  0  0  0  0  0  0  0  0

## data with sample size 20 without library read depth
res = simulate.data(seed = seed, numGroup0 = numGroup0, numGroup1 = numGroup1, sig0 = sig0, sig1= sig1, real.read.counts = real.read.counts, over.dispersion=NULL)
apply(res$data,1,sum)
## [1] 65 60 63 62 81 75 57 65 57 66 39 40 49 46 44 48 53 43 37 51
res$library.read.depth
## NULL

## data with sample size 20 without library read depth, but all samples in group 2 have zero read count.
res = simulate.data(seed = seed, numGroup0 = numGroup0, numGroup1 = numGroup1, sig0 = sig0, sig1= rep(0, length(sig1)), real.read.counts = real.read.counts, over.dispersion=NULL)
apply(res$data,1,sum)
## [1] 65 60 63 62 81 75 57 65 57 66  0  0  0  0  0  0  0  0  0  0
res$library.read.depth
##NULL
\end{lstlisting}
% Emacs 24.3.1 (Org mode 8.2.10)
\end{document}
